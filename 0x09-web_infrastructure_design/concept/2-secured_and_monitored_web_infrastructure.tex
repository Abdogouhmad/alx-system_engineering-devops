\documentclass[a4paper, 10pt]{article}
\usepackage{graphicx} % Required for inserting images
%styling
\linespread{1.5}
\usepackage[none]{hyphenat} %hyphens
\usepackage{indentfirst} %indent 1st lines
\usepackage{xcolor} %coloring text 
\usepackage{hyperref} %linkes
\usepackage{geometry}
\geometry{top=2cm,bottom=2cm,left=2.5cm,right=2.5cm}
\usepackage{fontspec}
\usepackage[document]{ragged2e}
\setmainfont{Arial}
\usepackage{enumerate}
\newcommand{\mdate}{July 2023}
\renewcommand\contentsname{Summary:}
\usepackage{float}
\usepackage{lipsum} %dummy text generated
\usepackage{natbib} %bibliography
\usepackage{array} %table
% EO styling

\begin{document}
\justifying
    \begin{center}
        ALX PROJECT \\
        Web infrastructure Design 
    \end{center}
\section*{\underline{Task 2:}}
    \textbf{\underline{Definitions and Explanations:}}
    \subsection*{For every additional element, why you are adding it}
        we added three new components to the infrastructure:
        a firewall for each server to protect them from any outcaste attack, 1 \textbf{SSL} certificate to 
        secure the connection between the client and the server, and three monitoring tools to monitor the servers and
        to collect logs and send to our data collector server. In addition, load balancer to distribute the traffic between the servers.
    \subsection*{What are firewalls for: }
        A firewall is a network security device that monitors incoming and outgoing network traffic and decides whether to allow or block specific traffic based on a defined set of security rules.
        Firewalls have been a first line of defense in network security for over 25 years. They establish a barrier between secured and controlled internal networks that can be trusted and untrusted outside networks, such as the Internet.
        A firewall can be hardware, software, or both.
    \subsection*{Why is the traffic served over HTTPS: }
        Because previously the traffic was served over HTTP, which is not secure,
        and we need to secure the traffic between the client and the server.
    \subsection*{What monitoring is used for: }
        Monitoring is the process of gathering metrics to determine the health and status of the 
        environment. Monitoring tools are used to detect problems in the infrastructure and 
        applications as soon as possible. Monitoring tools are also used to measure the performance 
        of the infrastructure and applications.
    \subsection*{How the monitoring tool is collecting data: }
        The monitoring tool is collecting data by sending requests to the servers and collecting the 
        response time, and the status code of the response. The monitoring tool is also collecting 
        logs from the servers and sending them to the data collector server.
    \subsection*{Explain what to do if you want to monitor your web server QPS: }
        To monitor the web server QPS, we need to use a monitoring tool that can send requests to the 
        server and collect the response time and the status code of the response. We can also use a 
        monitoring tool that can collect logs from the server and send them to the data collector 
        server. We can also use a monitoring tool that can collect metrics from the server and send 
        them to the data collector server.
\section*{Issues: }
    \subsection*{Why terminating SSL at the load balancer level is an issue: }
        it is an issue because decrypting the traffic at the load balancer level will increase the
        load on the load balancer, and it will also increase the latency of the requests. It will also
        increase the load on the servers because the servers will have to decrypt the traffic.
    \subsection*{Why having only one MySQL server capable of accepting writes is an issue: }
        because once it is down it means do data can be added or updated meaning some
        features of the application won’t work
    \subsection*{Why having servers with all the same components (database, web server and
    application server) might be a problem: }
    this is because once you have a bug in one of
    the components in one of the servers then the bug will be valid in the other servers.
\end{document}