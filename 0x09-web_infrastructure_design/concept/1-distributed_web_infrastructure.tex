\documentclass[a4paper, 10pt]{article}
\usepackage{graphicx} % Required for inserting images
%styling
\linespread{1.5}
\usepackage[none]{hyphenat} %hyphens
\usepackage{indentfirst} %indent 1st lines
\usepackage{xcolor} %coloring text 
\usepackage{hyperref} %linkes
\usepackage{geometry}
\geometry{top=2cm,bottom=2cm,left=2.5cm,right=2.5cm}
\usepackage{fontspec}
\usepackage[document]{ragged2e}
\setmainfont{Arial}
\usepackage{enumerate}
\newcommand{\mdate}{July 2023}
\renewcommand\contentsname{Summary:}
\usepackage{float}
\usepackage{lipsum} %dummy text generated
\usepackage{natbib} %bibliography
\usepackage{array} %table
% EO styling

\begin{document}
\justifying
    \begin{center}
        ALX PROJECT \\
        Web infrastructure Design 
    \end{center}
\section*{\underline{Task 1:}}
    \large
    \textbf{\underline{ Definitions and Explanations}}
    \subsection*{For every additional element, why you are adding it}
        We added a load balancer to distribute the workload across multiple servers. 
        This allows us to scale horizontally by adding more servers to the pool and 
        distribute the load among them. 
        It also ensures no single server is overwhelmed by requests and 
        prevents any single point of failure, improving reliability.
    \subsection*{What distribution algorithm your load balancer is configured with and how it works}
        Our load balancer utilizes the Round Robin algorithm, which connects in a sequential order unless a server becomes inaccessible.
        Requests are serviced sequentially, moving from one server to the next. Once all servers have received requests,
        the cycle recommences from the initial server.
        This approach is particularly effective when servers share equal specifications and there's minimal reliance on persistent connections.
    \subsection*{Is your load-balancer enabling an Active-Active or Active-Passive setup, explain the difference between both}
        Our load balancer is enabling an Active-Active setup. 
        In an Active-Active setup, all servers are active and capable of handling requests. 
        In contrast, in an Active-Passive setup, only one server is active and capable of handling 
        requests while the other servers are passive and only become active when the active server fails.
    \subsection*{How a database Primary-Replica (Master-Slave) cluster works}
        A Primary-Replica configures one server to act as the Primary server and the other server to act as a Replica of the Primary server. 
        However, the Primary server is capable of performing read or write requests while the Replica server is only capable of performing read requests.
        Data is synchronized between the Primary and Replica servers whenever the Primary server executes a write operation.
    \subsection*{What is the difference between the Primary node and the Replica node in regard to the application}
        In an application's distributed system, the primary node serves as the central source of truth, 
        managing write operations for data updates and maintaining consistency. It ensures data durability and handles failover situations. 
        In contrast, replica nodes store copies of data from the primary node, 
        allowing them to handle read operations efficiently, enhance availability, and contribute to overall system resilience by providing redundancy.
\section*{Issues: }
    \subsection*{SPOF}
        There are multiple SPOF (Single Point Of Failure).
        For example, if the Primary MySQL database server is down, 
        the entire site would be unable to make changes to the site (including adding or removing users).
        The server containing the load balancer and the application server connecting to the primary database server are also SPOFs.
    \subsection*{Security issues (no firewall, no HTTPS)}
        The data transmitted over the network isn't encrypted using an SSL certificate so hackers 
        can spy on the network. There is no way of blocking unauthorized IPs since there's no 
        firewall installed on any server.
    \subsection*{No monitoring}
        We have no way of knowing the status of
        each server since they're not being monitored.
\end{document}